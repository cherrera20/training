\documentclass[pdftex,hyperref]{beamer}

\usepackage[spanish]{babel}
\usepackage[T1]{fontenc}
\usepackage[latin9]{inputenc}

\usepackage{graphicx}
\usepackage{url}
\usepackage{hyperref}
\usepackage{amssymb}
\usepackage{colortbl}
\usepackage{listings}
\usepackage{fancyvrb}
\usepackage{multirow}
\usepackage{hyperref}
\usepackage{bibentry}
\usepackage{listings}
\usepackage{tabularx}

\definecolor{linkblue}{RGB}{49,57,174}
\definecolor{dkgreen}{rgb}{0,0.6,0}
\definecolor{gray}{rgb}{0.5,0.5,0.5}
\definecolor{mauve}{rgb}{0.58,0,0.82}

\lstdefinelanguage{scala}{
  morekeywords={abstract,annotation,case,catch,class,def,%
    do,else,extends,false,final,finally,%
    for,if,implicit,import,match,mixin,%
    new,null,object,override,package,%
    private,protected,requires,return,sealed,%
    super,this,throw,trait,true,try,%
    type,val,var,while,with,yield,
    macro},
  sensitive=true,
  morecomment=[l]{//},
  morecomment=[n]{/*}{*/},
  morestring=[b]",
  morestring=[b]',
  morestring=[b]"""
}
\lstset{frame=tb,
  language=scala,
  aboveskip=3mm,
  belowskip=3mm,
  showstringspaces=false,
  columns=flexible,
  basicstyle={\small\ttfamily},
  numbers=none,
  numberstyle=\tiny\color{gray},
  keywordstyle=\color{blue},
  commentstyle=\color{dkgreen},
  stringstyle=\color{mauve},
  frame=single,
  breaklines=true,
  breakatwhitespace=true
  tabsize=3
}

% Configuracion del documento PDF.
\hypersetup{
  pdfcreator=Jes�s L�pez Gonz�lez,
  backref%,
  %%pdfpagemode=FullScreen
}

% Configuracion pagina principal

\title{\textbf{Programaci�n Funcional en Scala}}
\subtitle{\textbf{ -- Tema 6 -- \\ Introducci�n a Play }}
\author[Jes�s L�pez Gonz�lez]{Jes�s L�pez Gonz�lez\\jesus.lopez@hablapps.com}
\institute[@jeslg]{Programaci�n Funcional en Scala\\ Habla Computing}
\date{Cursos ETSII-URJC 2015}

% Eleccion estilo de la presentacion

\mode<presentation>
{
 \usetheme{Madrid}
 \setbeamercovered{transparent}
}

\def\newblock{\hskip .11em plus .33em minus .07em}

% Configuracion del logo de la imagen

\subject{Talks}

\pgfdeclareimage[height=0.5cm]{university-logo}{images/logoURJC}
\logo{\pgfuseimage{university-logo}}

\setcounter{tocdepth}{1}

%NOANIMACION
\beamerdefaultoverlayspecification{}

% Volver a recordar tabla de contenidos en subsecciones

\AtBeginSection[]
{
  \begin{frame}<beamer>{�ndice}
    \tableofcontents[currentsection]
  \end{frame}
}

\nobibliography* 

\begin{document}

\begin{frame}
  \titlepage
\end{frame}

\section{Primeros Pasos con Play}

\begin{frame}
  \frametitle{Primeros Pasos con Play}

  \begin{block}{�Por qu� veremos Play en este curso?}
    \begin{itemize}
    \item Principal estandarte de Typesafe
    \item Paradigma Funcional llevado a un caso ``muy'' real
    \item El dominio del desarrollo web es ampliamente conocido
    \end{itemize}
  \end{block}
\end{frame}

\begin{frame}
  \frametitle{Primeros Pasos con Play}

  \begin{block}{�Qu� es Play?}
    \emph{Play!} es un framework web desarrollado por Guillaume Bort
    en el a�o 2007. Algunas de sus principales caracter�sticas son:

    \begin{itemize}
    \item Versiones Java y \textbf{Scala}
    \item \emph{Stateless}, en el sentido en que no se guarda
      informaci�n asociada a una sesi�n entre diferentes invocaciones
      de un mismo cliente.
    \item As�ncrono, extremadamente as�ncrono. Se usa el tipo de datos
      \emph{Future} como elemento c�ntrico para lograr tal feature.
    \item MVC, haremos especial �nfasis en el modelo y el controlador.
    \item {HTTP Awareness, no se oculta la naturaleza ``web'' del
      framework}
    \item \textbf{Framework Impuro} en cuanto al paradigma funcional.
    \end{itemize}
  \end{block}
\end{frame}

\begin{frame}
  \frametitle{Primeros Pasos con Play}

  \begin{block}{Instalaci�n del framework (requiere JDK instalado)}

    \begin{enumerate}
    \item Descargamos \emph{activator} desde el siguiente enlace
      \href{https://typesafe.com/get-started}{https://typesafe.com/get-started}
    \item Ejecutamos el comando \emph{activator new} para crear un
      nuevo proyecto a partir de las plantillas disponibles.
    \item Eligiremos la template \emph{play-scala} entre las opciones
      disponibles.
    \item Para nuestro primer proyecto, utilizaremos el nombre por
      defecto \emph{play-scala}.
    \item Observaremos que se ha creado un directorio
      \emph{play-scala} donde queda recogida la estructura de un
      proyecto Play.
    \item Mediante la orden \emph{activator run} (tras haber hecho un
      \emph{cd} al nuevo directorio creado) arrancaremos la
      aplicaci�n.
    \item Accediendo a
      \href{http://localhost:9000}{http://localhost:9000} desde
      nuestro navegador, deber�amos ver la p�gina de bienvenida.
    \end{enumerate}
  \end{block}
\end{frame}


\section{Estructura de un proyecto Play}

\begin{frame}
  \frametitle{Estructura de un proyecto Play}

  \begin{block}{Navegando por un proyecto Play}
    Aunque no es dif�cil darse cuenta de que estamos ante un proyecto
    \emph{sbt}, existen ciertos directorios que son particulares a
    Play. Se pretende navegar por la jerarqu�a de ficheros para
    identificar los elementos que componen una aplicaci�n desarrollada
    bajo este framework web:
    \begin{itemize}
    \item[conf] Aspectos de configuraci�n del framework (routing, logging\ldots)
    \item[app] Fuentes asociados a nuestra aplicaci�n (modelos,
      vistas, controladores\ldots)
    \item[test] Fuentes asociados a las pruebas de nuestra aplicaci�n
    \end{itemize}
  \end{block}
\end{frame}

\section{Pr�ctica Final}

\begin{frame}
  \frametitle{Pr�ctica Final}

  \begin{block}{Objetivo}
    Se pretende realizar una aplicaci�n web \emph{from scratch}
    poniendo en pr�ctica todo lo que se ha aprendido a lo largo del
    curso. Partimos de un lenguaje impuro (Scala) en un contexto
    impuro (la Web). �Hasta que punto es posible aplicar el paradigma
    funcional en semejante escenario?
  \end{block}
\end{frame}

\begin{frame}
  \frametitle{Pr�ctica Final}

  \begin{block}{Dominio}
    La aplicaci�n a realizar consistir� en un sencillo diccionario, en
    el que se puedan consultar/a�adir/eliminar entradas, siempre y
    cuando el usuario que realiza la petici�n correspondiente disponga
    de los permisos necesarios para poder llevar a cabo semejante
    acci�n.
  \end{block}
\end{frame}


\section{Takeaways}

\begin{frame}
  \frametitle{Takeaways}
\end{frame}


\end{document} 
