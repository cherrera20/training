
\documentclass[pdftex,hyperref]{beamer}


\usepackage[spanish]{babel}
\usepackage[T1]{fontenc}
\usepackage[latin9]{inputenc}

%\usepackage{cite}
\usepackage{graphicx}
\usepackage{url}
\usepackage{hyperref}
\usepackage{amssymb}
\usepackage{colortbl}
\usepackage{listings}
\usepackage{fancyvrb}
\usepackage{multirow}
\usepackage{hyperref}
\usepackage{bibentry}
\usepackage{listings}
\usepackage{tabularx}



% Configuracion del documento PDF.
\hypersetup{
  pdfcreator=Jes�s L�pez Gonz�lez,
  backref,
  pdfpagemode=FullScreen
}





% Configuracion pagina principal

\title{\textbf{Programaci�n Funcional en Scala}}
\subtitle{\textbf{ -- Tema 0 -- \\ Introducci�n y Objetivos del Curso}}
\author[Jes�s L�pez Gonz�lez]{Jes�s L�pez Gonz�lez\\jesus.lopez@hablapps.com}
\institute[@jeslg]{Programaci�n Funcional en Scala\\ Habla Computing}
\date{Cursos ETSII-URJC 2015}




% Eleccion estilo de la presentacion

\mode<presentation>
{
 \usetheme{Madrid}
% \usetheme{Antibes}
 \setbeamercovered{transparent}
}


\def\newblock{\hskip .11em plus .33em minus .07em}


% Configuracion del logo de la imagen

\subject{Talks}

\pgfdeclareimage[height=0.5cm]{university-logo}{images/logoURJC}
\logo{\pgfuseimage{university-logo}}


\setcounter{tocdepth}{1}

%NOANIMACION
\beamerdefaultoverlayspecification{}

% Volver a recordar tabla de contenidos en subsecciones

\AtBeginSection[]
{
  \begin{frame}<beamer>{�ndice}
    \tableofcontents[currentsection]
  \end{frame}
}

\nobibliography* 

\begin{document}

\begin{frame}
  \titlepage
\end{frame}

\section{Presentaci�n} 

\begin{frame}  
  \begin{block}{Presentaci�n}
    Jes�s L�pez Gonz�lez
    \begin{itemize}
    \item Correo-e: \href{mailto:jesus.lopez@hablapps.com}{jesus.lopez@hablapps.com}
    \item Twitter: \href{http://twitter.com/jeslg}{@jeslg}
    \item \href{http://hablapps.com}{Habla Computing}
    \item \href{http://www.meetup.com/Scala-Programming-Madrid/}{Scala Programming @ Madrid}
    \end{itemize}
  \end{block}
\end{frame}

\section{Informaci�n del Curso} 

\begin{frame}  
  \begin{block}{Informaci�n del Curso}
    Programaci�n Funcional en Scala
    \begin{itemize}
    \item Horario: \alert{L-X-V} (17h-20h)
    \item Fechas: 9, 11, 16, 18, 23, 25, 27 de marzo, 8, 10 y 13 de abril
    \item Duraci�n: 30 horas
    \item Cr�ditos: 1.5 cr�ditos ECTS
    \item Repositorio Oficial:
      \href{http://github.com/hablapps/training/}{http://github.com/hablapps/training/}
    \end{itemize}
  \end{block}
\end{frame}

\section{Temario}

\begin{frame}  
  \begin{block}{Temario}
    Programaci�n Funcional en Scala
    \begin{itemize}
    \item[T1] Introducci�n a Scala (9 de marzo)
    \item[T2] Introducci�n a la Programaci�n Funcional (11 y 16 de marzo)
    \item[T3] Constructores de Tipos, Type Classes y Funtores (18 de marzo)
    \item[T4] �No tengas miedo de las m�nadas! (23 y 25 de marzo)
    \item[T5] Efectos, EDSLs e Int�rpretaci�n (27 de marzo)
    \item[T6] Introducci�n a Play (8 de abril)
    \item[T7] Desplegando Servicios en Play (10 de abril)
    \item[Fin] \emph{Scala @ Real Life} (13 de abril)
    \end{itemize}
  \end{block}
\end{frame}

\section{Objetivos}

\begin{frame}  
  \begin{block}{Objetivos Principales}
    \begin{itemize}
    \item Entender los fundamentos del paradigma de la Programaci�n Funcional
    \item Identificar y erradicar los \alert{efectos de lado} de un programa
    \item Aplicar la separaci�n de aspectos: efectos e interpretaci�n
    \end{itemize}
  \end{block}
  \begin{block}{Y de forma indirecta...}
    \begin{itemize}
    \item Aprender los fundamentos de la programaci�n en \emph{Scala}
    \item Poder desenvolverse en el ecosistema Scala (REPL, SBT, ScalaTest...)
    \item Conocer las nociones b�sicas del framework web \emph{Play}
    \end{itemize}
  \end{block}
\end{frame}

\section{Referencias}
\begin{frame}  
  \begin{block}{Referencias}
    \begin{itemize}
    \item Chiusano, P. y Bjarnason, R. (2014) \emph{Functional Programming in Scala}
    \item Lipovaca, M. (2011) \emph{Learn You a Haskell for Great Good!}
    \item Odersky, M. (2011) \emph{Programming in Scala, 2nd Edition}
    \item Gonz�lez, G. \href{http://www.haskellforall.com}{\emph{Haskell For All}} (blog)
    \end{itemize}
  \end{block}
\end{frame}


\end{document}





