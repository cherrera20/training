\documentclass[pdftex,hyperref]{beamer}

\usepackage[spanish]{babel}
\usepackage[T1]{fontenc}
\usepackage[latin9]{inputenc}

\usepackage{graphicx}
\usepackage{url}
\usepackage{hyperref}
\usepackage{amssymb}
\usepackage{colortbl}
\usepackage{listings}
\usepackage{fancyvrb}
\usepackage{multirow}
\usepackage{hyperref}
\usepackage{bibentry}
\usepackage{listings}
\usepackage{tabularx}

\definecolor{linkblue}{RGB}{49,57,174}
\definecolor{dkgreen}{rgb}{0,0.6,0}
\definecolor{gray}{rgb}{0.5,0.5,0.5}
\definecolor{mauve}{rgb}{0.58,0,0.82}

\lstdefinelanguage{scala}{
  morekeywords={abstract,annotation,case,catch,class,def,%
    do,else,extends,false,final,finally,%
    for,if,implicit,import,match,mixin,%
    new,null,object,override,package,%
    private,protected,requires,return,sealed,%
    super,this,throw,trait,true,try,%
    type,val,var,while,with,yield,
    macro},
  sensitive=true,
  morecomment=[l]{//},
  morecomment=[n]{/*}{*/},
  morestring=[b]",
  morestring=[b]',
  morestring=[b]"""
}
\lstset{frame=tb,
  language=scala,
  aboveskip=3mm,
  belowskip=3mm,
  showstringspaces=false,
  columns=flexible,
  basicstyle={\small\ttfamily},
  numbers=none,
  numberstyle=\tiny\color{gray},
  keywordstyle=\color{blue},
  commentstyle=\color{dkgreen},
  stringstyle=\color{mauve},
  frame=single,
  breaklines=true,
  breakatwhitespace=true
  tabsize=3
}

% Configuracion del documento PDF.
\hypersetup{
  pdfcreator=Jes�s L�pez Gonz�lez,
  backref%,
  %%pdfpagemode=FullScreen
}

% Configuracion pagina principal

\title{\textbf{Programaci�n Funcional en Scala}}
\subtitle{\textbf{ -- Tema 3 -- \\ Constructores de Tipos, Typeclasses y Funtores }}
\author[Jes�s L�pez Gonz�lez]{Jes�s L�pez Gonz�lez\\jesus.lopez@hablapps.com}
\institute[@jeslg]{Programaci�n Funcional en Scala\\ Habla Computing}
\date{Cursos ETSII-URJC 2015}

% Eleccion estilo de la presentacion

\mode<presentation>
{
 \usetheme{Madrid}
 \setbeamercovered{transparent}
}

\def\newblock{\hskip .11em plus .33em minus .07em}

% Configuracion del logo de la imagen

\subject{Talks}

\pgfdeclareimage[height=0.5cm]{university-logo}{images/logoURJC}
\logo{\pgfuseimage{university-logo}}

\setcounter{tocdepth}{1}

%NOANIMACION
\beamerdefaultoverlayspecification{}

% Volver a recordar tabla de contenidos en subsecciones

\AtBeginSection[]
{
  \begin{frame}<beamer>{�ndice}
    \tableofcontents[currentsection]
  \end{frame}
}

\nobibliography* 

\begin{document}

\begin{frame}
  \titlepage
\end{frame}

\section{Higher Kinded Types}

\begin{frame}[fragile]
  \frametitle{Higher Kinded Types}

  \begin{block}{Constructor de Tipos}
    Un constructor de tipos es un tipo que recibe argumentos tipo para
    construir un nuevo tipo. Ejemplos de constructores de tipo pueden
    ser \emph{List} (recibe un argumento tipo), \emph{Option} (recibe
    un argumento tipo) o \emph{Map} (recibe dos argumentos tipo).
  \end{block}

  %% Option no puede usarse como tal, se requieren par�metros tipo.

  \begin{lstlisting}[language=scala, caption=Usando constructores de tipo]
scala> val opt: Option[Int] = Some(3)
opt: Option[Int] = Some(3)

scala> val lst: List[String] = List("how", "are", "you")
lst: List[String] = List(how, are, you)

scala> val map: Map[Int, String] = Map(1 -> "one", 2 -> "two")
map: Map[Int,String] = Map(1 -> one, 2 -> two)
  \end{lstlisting}
\end{frame}

\begin{frame}[fragile]
  \frametitle{Higher Kinded Types}

  \begin{block}{Kind}
    El kind es el ``tipo'' de un constructor de tipos.
    \begin{itemize}
      \item Proper: \emph{(*)} Ej: String, Int, List[Int], etc.
      \item First-order: \emph{(* -> *)} Ej: Option[\_], List[\_], etc.
      \item Higher-order: \emph{((* -> *) -> *)} Ej: Functor[F[\_]], etc.
    \end{itemize}
  \end{block}
\end{frame}

\begin{frame}[fragile]
  \frametitle{Higher Kinded Types}

  \begin{block}{Ejercicio: �Qu� kind tienen los siguientes tipos?}
    \begin{itemize}
      \item Int
      \item Option[Int]
      \item List
      \item Map
      \item Map[Int, \_]
      \item Monad[M[\_]]
    \end{itemize}
  \end{block}
\end{frame}

\begin{frame}[fragile]
  \frametitle{Higher Kinded Types}

  %% Mostrar un ejemplo con Tuple2

  \begin{lstlisting}[language=scala, caption=Funci�n show recibe un
      constructor de tipos C (* -> * -> *)]
scala> def show[C[_, _], A, B](arg: C[A, B]): String = arg.toString
show: [C[_, _], A, B](arg: C[A,B])String

scala> show(Map(1 -> "one", 2 -> "two"))
res0: String = Map(1 -> one, 2 -> two)

scala> show(List(1, 2, 3))
<console>:9: error: inferred kinds of the type arguments (scala.collection.LinearSeqOptimized,Int,List[Int]) do not conform to the expected kinds of the type parameters (type C,type A,type B).
  \end{lstlisting}
\end{frame}

\section{Implicits}

\begin{frame}
  \frametitle{Implicits}

  \begin{block}{Intuici�n}
    Por norma general, los programadores trabajamos de forma
    expl�cita. Cuando queremos aplicar una funci�n indicamos de forma
    muy clara cu�les son los par�metros de entrada. Adem�s, cuando
    queremos invocar una funci�n, dejamos plasmada nuestra intenci�n
    de hacerlo en el c�digo. Scala trae consigo varias t�cnicas para
    trabajar de forma impl�cita, que permiten al compilador tomar
    decisiones de forma aut�noma, siempre tratando de intuir las
    intenciones del programador. Podemos diferenciar tres grupos:
    \begin{itemize}
      \item Implicit Parameters
      \item Implicit Defs
      \item Implicit Classes
    \end{itemize}
  \end{block}
\end{frame}

\begin{frame}[fragile]
  \frametitle{Implicits}

  \begin{lstlisting}[language=scala, caption=Implicit Parameters]
scala> def add(a: Int)(implicit b: Int): Int = a + b
add: (a: Int)(implicit b: Int)Int

scala> add(1)(2)
res2: Int = 3

scala> implicit val x: Int = 2
x: Int = 2

scala> add(1)
res3: Int = 3
  \end{lstlisting}
\end{frame}


\begin{frame}[fragile]
  \frametitle{Implicits}

  %% S�lo trabajan con un nivel de conversi�n

  \begin{lstlisting}[language=scala, caption=Implicit Defs]
scala> case class Point(x: Int, y: Int)
defined class Point

scala> implicit def tupleToPoint(tuple: (Int, Int)): Point =
     |   Point(tuple._1, tuple._2)
tupleToPoint: (tuple: (Int, Int))Point

scala> val p1: Point = tupleToPoint((1, 2))
p1: Point = Point(1,2)

scala> val p2: Point = (1, 2)
p2: Point = Point(1,2)
  \end{lstlisting}
\end{frame}

\begin{frame}[fragile]
  \frametitle{Implicits}

  \begin{lstlisting}[language=scala, caption=Implicit Classes]
scala> implicit class IntExtender(val i: Int) {
     |   def myNewDef: String = "Def not implemented for Int"
     | }
defined class IntExtender

scala> 33.myNewDef
res0: String = Def not implemented for Int
  \end{lstlisting}
\end{frame}

\begin{frame}[fragile]
  \frametitle{Implicits}

  \begin{block}{Impl�citos y Constructores de Tipos}
    Existe una notaci�n especial para lidiar con constructores de
    tipos e impl�citos bajo una sintaxis m�s dulcificada. Tal notaci�n
    es conocida como \emph{context bound} y es muy �til para proveer a
    la funci�n de ciertas evidencias que resultan necesarias para su
    aplicaci�n.
  \end{block}
\end{frame}

\begin{frame}[fragile]
  \frametitle{Implicits}

  \begin{lstlisting}[language=scala, caption=Dulcificaci�n de
      evidencias impl�citas en serialize2]
scala> trait Serializer[A] {
     |   def serialize(a: A): String
     | }
defined trait Serializer

scala> def serialize[A](a: A)(implicit ev: Serializer[A]): String =
     |   ev.serialize(a)
serialize: [A](a: A)(implicit ev: Serializer[A])String

scala> def serialize2[A: Serializer](a: A): String =
     |   implicitly[Serializer[A]].serialize(a)
serialize2: [A](a: A)(implicit evidence$1: Serializer[A])String
  \end{lstlisting}
\end{frame}

\section{Typeclasses}

\begin{frame}[fragile]
  \frametitle{Typeclasses}

  \begin{block}{Typeclasses}
    \emph{``A typeclass is a sort of interface that defines some behavior. If
    a type is a part of a typeclass, that means that it supports and
    implements the behavior the typeclass describes. A lot of people
    coming from OOP get confused by typeclasses because they think
    they are like classes in object oriented languages. Well, they're
    not. You can think of them kind of as Java interfaces, only
    better.''} (Learn You a Haskell for Great Good)
  \end{block}
\end{frame}

\begin{frame}[fragile]
  \frametitle{Typeclasses}

  \begin{lstlisting}[language=scala, caption=Typeclass Eq]
trait Eq[A] {
  def eq(x: A, y: A): Boolean
}
  \end{lstlisting}
\end{frame}

\begin{frame}[fragile]
  \frametitle{Typeclasses}

  \begin{lstlisting}[language=scala, caption=Ejercicio: Implementar la
      Typeclass YesNo (javascript)]
trait YesNo[A] {
  def yesNo(x: A): Boolean
}
  \end{lstlisting}
\end{frame}

\begin{frame}[fragile]
  \frametitle{Typeclasses}

  %% Intuici�n de ``lift''

  \begin{lstlisting}[language=scala, caption=Typeclass Functor]
trait Functor[F[_]] {
  def map[A, B](f: A => B)(value: F[A]): F[B]
}
  \end{lstlisting}
\end{frame}

\section{Takeaways}

\begin{frame}
  \frametitle{Takeaways}

  \begin{block}{}
    Hemos descubierto el concepto \emph{kind} que nos permite conocer
    el tipo de un constructor de tipos.
  \end{block}

  \begin{block}{}
    Hemos trabajado con impl�citos, permitiendo al compilador el que
    pueda tomar ciertas decisiones por nosotros, para facilitar
    nuestro trabajo.
  \end{block}

  \begin{block}{}
    Hemos visto qu� es una \emph{typeclass} y c�mo se realiza su
    implementaci�n en Scala.
  \end{block}

  \begin{block}{}
    Sabemos qu� forma tiene un \emph{functor}. Desarrollaremos el
    concepto en m�s profundidad durante las pr�ximas clases.
  \end{block}
\end{frame}

\end{document}
